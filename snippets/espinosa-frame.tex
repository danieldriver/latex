% !TEX TS-program = XeLaTeX
% !TEX encoding = UTF-8 Unicode

\documentclass[11pt,twoside]{article}

% Page layout
\usepackage[body={28pc,56pc},marginparsep=1pc,marginparwidth=12pc,%showframe,
  includemp,hmarginratio=1:1,vmarginratio=1:2]{geometry}

% Font size and leading (defaults: http://en.wikibooks.org/wiki/LaTeX/Fonts#Sizing_text)
\renewcommand\tiny{\fontsize{7}{10}\selectfont}
\renewcommand\scriptsize{\fontsize{8}{11}\selectfont}
\renewcommand\footnotesize{\fontsize{9}{12}\selectfont}
\renewcommand\small{\fontsize{10}{13}\selectfont}
\renewcommand\normalsize{\fontsize{11}{13.6}\selectfont}% 11pt base size
\renewcommand\large{\fontsize{12}{14}\selectfont}
\renewcommand\Large{\fontsize{14}{16}\selectfont}
\renewcommand\LARGE{\fontsize{18}{20}\selectfont}
\renewcommand\huge{\fontsize{24}{26}\selectfont}
\renewcommand\Huge{\fontsize{30}{32}\selectfont}

% Basic fontspec setup
\usepackage{fontspec}
\defaultfontfeatures{Ligatures=TeX}
\setmainfont{Espinosa Nova}
\newfontface\O[StylisticSet=1]{EspinosaNova-Ornaments}

% Colo(u)r
\usepackage[x11names,hyperref]{xcolor}
\newcommand\red[1]{\color{Red3} #1}

\setlength\parindent{0pt}
\setlength\parskip{0pt}
\pagestyle{empty}

% the picture package allows use of dimens in the arguments of picture commands
\usepackage{picture}

\begin{document}

% assume 11pt base rather than calculating for any font size
\newlength\bskip
\newlength\bpad
\newlength\bheight
\setlength\bskip{10.96pt}
\setlength\bpad{1.892pt}
\setlength\bheight{6\bskip}

% full crenellation
\begin{picture}(\textwidth,\bheight+\bskip-2\bpad)\O
\put(-\bpad,0){7}% draw bottom border
\multiput(\bskip-\bpad,0)(\bskip,0){29}{Z}
\put(30\bskip-\bpad,0){9}
\put(-\bpad,\bheight){1}% draw top border
\multiput(\bskip-\bpad,\bheight)(\bskip,0){29}{Y}
\put(30\bskip-\bpad,\bheight){3}
\multiput(-\bpad,\bskip)(0,\bskip){5}{y}% draw left border
\multiput(30\bskip-\bpad,\bskip)(0,\bskip){5}{z}% draw right border
\end{picture}

\medskip

% half crenellation
\begin{picture}(\textwidth,\bheight+\bskip-2\bpad)\O
\put(-\bpad,0){7}% draw bottom border
\multiput(\bskip-\bpad,0)(2\bskip,0){14}{Z8}
\put(29\bskip-\bpad,0){Z}
\put(30\bskip-\bpad,0){9}
\put(-\bpad,\bheight){1}% draw top border
\multiput(\bskip-\bpad,\bheight)(2\bskip,0){14}{Y2}
\put(29\bskip-\bpad,\bheight){Y}
\put(30\bskip-\bpad,\bheight){3}
\multiput(-\bpad,\bskip)(0,2\bskip){3}{y}% draw left border
\multiput(-\bpad,2\bskip)(0,2\bskip){2}{4}
\multiput(30\bskip-\bpad,\bskip)(0,2\bskip){3}{z}% draw right border
\multiput(30\bskip-\bpad,2\bskip)(0,2\bskip){2}{6}
\end{picture}

\end{document}  
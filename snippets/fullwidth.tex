%!TEX TS-program = XeLaTeX
%!TEX encoding = UTF-8 Unicode

\documentclass[twoside]{article}

% Minimal fontspec definitions
\usepackage{fontspec}
\setmainfont[Ligatures=TeX]{Minion Pro}
\setsansfont[Ligatures=TeX]{Scala Sans Pro}
\setmonofont[Scale=MatchLowercase]{Nitti}
\frenchspacing

% Set captions in sans serif (and sans package; see §3 in `texdoc ccaption`)
\makeatletter
\renewcommand{\fnum@figure}[1]{\sffamily\footnotesize\textbf{\figurename~\thefigure}.}
\renewcommand{\fnum@table}[1]{\sffamily\footnotesize\textbf{\tablename~\thetable}.}
% Remove caption skip and set them ragged (cf /usr/local/texlive/2012/texmf-dist/source/latex/base/classes.dtx)
\renewcommand{\@makecaption}[2]{\raggedright#1: #2\par}
\makeatother
% Marginalize figure and table captions using this fork of Stephan Hennig's `mcaption` package
% (v3.0 [2009/03/13], cf `texdoc mcaption` and /usr/local/texlive/2012/texmf-dist/tex/latex/mcaption/mcaption.sty)
\usepackage[strict]{changepage}
\makeatletter
% 1. Define initial commands and storage boxes
\newcommand*{\marginalize@CaptionLong}{}
\newcommand*{\marginalize@CaptionShort}{}
\newcommand*{\marginalize@CaptionFlag}{}
\newcommand*{\marginalize@Label}{}
\newcommand*{\marginalize@LabelFlag}{}
\newsavebox{\marginalize@ObjectBox}
\newsavebox{\marginalize@CaptionBox}
% 2. Make marginalize environment
\newenvironment{marginalize}{% collect the \caption and \label commands
  \let\marginalize@origcaption\caption%
  \let\caption\marginalize@caption%
  \gdef\marginalize@CaptionFlag{f}%
  \let\marginalize@origlabel\label%
  \let\label\marginalize@label%
  \gdef\marginalize@LabelFlag{f}%
  \begin{lrbox}{\marginalize@ObjectBox}% place environment contents in a box
    \begin{minipage}{\linewidth}%
}{%
    \end{minipage}%
  \end{lrbox}%
  \let\caption\marginalize@origcaption% restore original caption and label
  \let\label\marginalize@origlabel%
  \marginalize@align@boxes% ship out results for processing
  \marginalize@output@boxes%
}%
% 3. Define auxillary storage commands
\newcommand*{\marginalize@origcaption}{}% store original commands
\newcommand*{\marginalize@origlabel}{}
\newcommand*{\marginalize@caption}[2][]{% extract and flag the caption
  \gdef\marginalize@CaptionShort{#1}%
  \gdef\marginalize@CaptionLong{#2}%
  \gdef\marginalize@CaptionFlag{t}%
  \ignorespaces
}%
\newcommand*{\marginalize@label}[1]{% extract and flag the label
  \gdef\marginalize@Label{#1}%
  \gdef\marginalize@LabelFlag{t}%
  \ignorespaces
}%
% 4. Prepare and align contents and caption boxes
\newcommand*{\marginalize@align@boxes}{%
  \begin{lrbox}{\marginalize@CaptionBox}% process caption or make an empty box
    \begin{minipage}{\marginparwidth}%
      \if\marginalize@CaptionFlag t%
        \caption[\marginalize@CaptionShort]{\strut\marginalize@CaptionLong\strut}
      \fi%
      \if\marginalize@LabelFlag t%
        \label{\marginalize@Label}%
      \fi%
    \end{minipage}%
  \end{lrbox}%
  % Wrap contents in zero height lines and use \vtop to peg alignment (alt: \vbox for bottom)
  % TK -- could set ObjectBox to \vbox to get caption below it, e.g. for a full-width * variation
  \sbox{\marginalize@ObjectBox}{\vtop{%
    \vskip0pt%
    \hbox{\usebox{\marginalize@ObjectBox}}%
    \vskip0pt}%
  }%
  \sbox{\marginalize@CaptionBox}{\vtop{%
    \vskip0pt%
    \hbox{\usebox{\marginalize@CaptionBox}}%
    \vskip0pt}%
  }%
}%
% 5. Output the contents and the caption box
\newcommand*{\marginalize@output@oddpage}{% for odd pages
  \makebox[\linewidth][l]{%
    \usebox{\marginalize@ObjectBox}%
    \hspace*{\marginparsep}%
    \smash{\usebox{\marginalize@CaptionBox}}%
  }%
}%
\newcommand*{\marginalize@output@evenpage}{% for even pages
  \makebox[\linewidth][r]{%
    \smash{\usebox{\marginalize@CaptionBox}}%
    \hspace*{\marginparsep}%
    \usebox{\marginalize@ObjectBox}%
  }%
}%
\newcommand*{\marginalize@output@boxes}{% finally, check the pages and build the appropriate layouts
  \if@twoside%
    \checkoddpage%
    \ifoddpage%
      \marginalize@output@oddpage%
    \else%
      \marginalize@output@evenpage%
    \fi%
  \else%
    \marginalize@output@oddpage%
  \fi%
}%
\makeatother
% End fork of mcaption package
% 
% 6. Bake marginalize environment into all figures and tables
\let\oldfigure\figure
\let\endoldfigure\endfigure
\renewenvironment{figure}[1][]{%
  \oldfigure[#1]\begin{marginalize}}{%
  \end{marginalize}\endoldfigure}
\let\oldtable\table
\let\endoldtable\endtable
\renewenvironment{table}[1][ht]{%
  \oldtable[#1]\begin{marginalize}}{%
  \end{marginalize}\endoldtable}
% add tabular? cf http://www.faqoverflow.com/tex/6033.html

% Call longtable and put its captions in the margin (cf http://tex.stackexchange.com/a/96025)
\usepackage{longtable}
\makeatletter
\def\LT@makecaption#1#2#3{%
  \LT@mcol\LT@cols c{\hbox to\z@{\hss\parbox[t]\LTcapwidth{% original definition in longtable.sty
    \sbox\@tempboxa{#1{#2: }#3}%
    \ifdim\wd\@tempboxa>\hsize
      #1{#2: }#3%
    \else
      \hbox to\hsize{\hfil\box\@tempboxa\hfil}%
    \fi
    \endgraf\vskip\baselineskip}%
  \hss}}}
\makeatother 

% Put footnotes in the margins as well. Note that footmisc needs hyperref links
% to notes switched off, eg: \usepackage[hyperfootnotes=false,...]{hyperref}
\usepackage[side,multiple]{footmisc}
\usepackage{marginfix}
% Place fn mark in paragraph, and mark as tables and figures (bold + period).
\makeatletter
  \renewcommand\@makefntext[1]{%
  \sffamily\raggedright\parindent=1em\noindent
  {}{\bfseries\@thefnmark.}~#1}%
\makeatother

% Packages to show what's happening
\usepackage{lipsum}
\usepackage{showframe}

\title{Brief Article}
\author{The Author}
%\date{}

\begin{document}
\maketitle

\lipsum[1-2]

\begin{figure}[htbp]
  \centering
  \includegraphics{./logo.png}
  \caption{An image and a caption}
\end{figure}

\lipsum[3-4]

\begin{figure}[htbp]
\centering
\includegraphics{./logo.png}
\caption{Same image again, in standard pandoc output. It comes from the \TeX\ Stack Exchange. It has such a long caption that it spills to two lines even in the main \texttt{textwidth} section.}
\end{figure}

\lipsum[5-7]

Here are short footnote references,\footnote{Here is the first footnote.}\footnote{Here is the second.} and another long one.

This paragraph isn't be part of the note, because it isn't indented in pandoc.

\begin{table}
    \centering
    \begin{tabular}{rrrrrrrrr}
        1 & 1 & 1 & 1 & 1 & 1 & 1 & 1 \\
        1 & 1 & 1 & 1 & 1 & 1 & 1 & 1
    \end{tabular}
    \caption{Caption of the normal \texttt{table} environment.}
    \label{table}
\end{table}

\begin{longtable}{rrrrrrrrr}
    1 & 1 & 1 & 1 & 1 & 1 & 1 & 1 \\
    1 & 1 & 1 & 1 & 1 & 1 & 1 & 1 \\
    \caption{Caption of the \texttt{longtable} environment.}
    \label{longtable}
\end{longtable}

\begin{longtable}[c]{@{}clrl@{}} % standard output from pandoc markdown input
\hline\noalign{\medskip}
\begin{minipage}[b]{0.17\columnwidth}\centering
Centered Header
\end{minipage} & \begin{minipage}[b]{0.11\columnwidth}\raggedright
Default Aligned
\end{minipage} & \begin{minipage}[b]{0.22\columnwidth}\raggedleft
Right Aligned
\end{minipage} & \begin{minipage}[b]{0.35\columnwidth}\raggedright
Left Aligned
\end{minipage}
\\\noalign{\medskip}
\hline\noalign{\medskip}
\begin{minipage}[t]{0.17\columnwidth}\centering
First
\end{minipage} & \begin{minipage}[t]{0.11\columnwidth}\raggedright
row
\end{minipage} & \begin{minipage}[t]{0.22\columnwidth}\raggedleft
12.0
\end{minipage} & \begin{minipage}[t]{0.35\columnwidth}\raggedright
Example of a row that spans multiple lines.
\end{minipage}
\\\noalign{\medskip}
\begin{minipage}[t]{0.17\columnwidth}\centering
Second
\end{minipage} & \begin{minipage}[t]{0.11\columnwidth}\raggedright
row
\end{minipage} & \begin{minipage}[t]{0.22\columnwidth}\raggedleft
5.0
\end{minipage} & \begin{minipage}[t]{0.35\columnwidth}\raggedright
Here's another one. Note the blank line between rows.
\end{minipage}
\\\noalign{\medskip}
\hline
\noalign{\medskip}
\caption{Here's the caption. In pandoc markdown it, too, may span multiple lines.}
\end{longtable}

\lipsum[1]

\end{document}
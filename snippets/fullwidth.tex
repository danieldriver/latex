%!TEX TS-program = xelatex
%!TEX encoding = UTF-8 Unicode

\documentclass[twoside]{article}

% Minimal fontspec definitions
\usepackage{fontspec}
\setmainfont[Ligatures=TeX]{Minion Pro}
\setsansfont[Ligatures=TeX]{Scala Sans Pro}
\setmonofont[Scale=MatchLowercase]{Nitti}
\frenchspacing

% Set captions in sans serif, ragged right, and in the margin
\makeatletter
\renewcommand{\fnum@figure}[1]{\sffamily\footnotesize\textbf{\figurename~\thefigure}.}
\renewcommand{\fnum@table}[1]{\sffamily\footnotesize\textbf{\tablename~\thetable}.}
\renewcommand{\@makecaption}[2]{\raggedright#1: #2\par}
\makeatother
% Call mcaption and bake into figure and table environs
\usepackage{mcaption}
\let\oldfigure\figure% for table and tabular environs cf http://www.faqoverflow.com/tex/6033.html
\let\endoldfigure\endfigure
\renewenvironment{figure}[1][]{%
  \oldfigure[#1]\begin{margincap}}{%
  \end{margincap}\endoldfigure}
% Call longtable and put its captions in the margin, for single or facing pages
\usepackage{longtable}
\usepackage[strict]{changepage}% provides \adjustwidth as well
\setlength\LTpost{0pt}
\makeatletter
\def\LT@makecaption#1#2#3{% cf http://tex.stackexchange.com/a/96025
  \noalign{\smash{\hbox{%
    \if@twoside
      \checkoddpage
      \ifoddpage
        \kern\textwidth\kern\marginparsep
      \else
        \kern-\marginparsep\kern-\marginparwidth
      \fi
    \else
      \kern\textwidth\kern\marginparsep
    \fi
  \rlap{\parbox[b]{\marginparwidth}{\raggedright#1{#2: }#3\medskip}}
  }}}%
}%
\makeatother
% Put footnotes in the margins as well -- ote that footmisc needs hyperref links
% to notes switched off, eg: \usepackage[hyperfootnotes=false,...]{hyperref}
\usepackage[side,multiple]{footmisc}
\usepackage{marginfix}
\makeatletter
  \renewcommand\@makefntext[1]{%
  \sffamily\raggedright\parindent=1em\noindent
  {}{\bfseries\@thefnmark.}~#1}%
\makeatother

% Create fullwidth environment
\makeatletter
\newenvironment{fullwidth}{%
  \begin{adjustwidth\if@twoside*\fi}{}{\dimexpr-\marginparsep-\marginparwidth\relax}}{%
  \end{adjustwidth\if@twoside*\fi}}
\makeatother

% Packages to show what's happening
\usepackage{lipsum}
%\usepackage{showframe}

\title{Brief Article}
\author{The Author}
%\date{}

\begin{document}
\maketitle

\lipsum[1-2]

\begin{figure}[htbp]
  \centering
  \includegraphics{./logo.png}
  \caption{An image and a caption}
\end{figure}

\lipsum[3-4]

\begin{figure}[htbp]
\centering
\includegraphics{./logo.png}
\caption{Same image again, in standard pandoc output. It comes from the \TeX\ Stack Exchange. It has such a long caption that it spills to two lines even in the main \texttt{textwidth} section.}
\end{figure}

\lipsum[5-6]

Here are short footnote references,\footnote{Here is the first footnote.}\footnote{Here is the second.} and another long one.%\footnote{Here's one with multiple paragraphs.
%
%  In pandoc markdown, subsequent paragraphs are indented to show
%  that they belong to the previous footnote.
%
%  The whole paragraph can be indented, or just the first line. In this
%  way, multi-paragraph footnotes work like multi-paragraph list items.}

This paragraph isn't be part of the note, because it isn't indented in pandoc.

\begin{longtable}{rrrrrrrrr}
    1 & 1 & 1 & 1 & 1 & 1 & 1 & 1 \\
    1 & 1 & 1 & 1 & 1 & 1 & 1 & 1 \\
    \caption{Caption of the \texttt{longtable} environment.}
    \label{longtable}
\end{longtable}

\lipsum[7-8]

\begin{longtable}[c]{@{}clrl@{}} % standard output from pandoc markdown input
\hline\noalign{\medskip}
\begin{minipage}[b]{0.17\columnwidth}\centering
Centered Header
\end{minipage} & \begin{minipage}[b]{0.11\columnwidth}\raggedright
Default Aligned
\end{minipage} & \begin{minipage}[b]{0.22\columnwidth}\raggedleft
Right Aligned
\end{minipage} & \begin{minipage}[b]{0.35\columnwidth}\raggedright
Left Aligned
\end{minipage}
\\\noalign{\medskip}
\hline\noalign{\medskip}
\begin{minipage}[t]{0.17\columnwidth}\centering
First
\end{minipage} & \begin{minipage}[t]{0.11\columnwidth}\raggedright
row
\end{minipage} & \begin{minipage}[t]{0.22\columnwidth}\raggedleft
12.0
\end{minipage} & \begin{minipage}[t]{0.35\columnwidth}\raggedright
Example of a row that spans multiple lines.
\end{minipage}
\\\noalign{\medskip}
\begin{minipage}[t]{0.17\columnwidth}\centering
Second
\end{minipage} & \begin{minipage}[t]{0.11\columnwidth}\raggedright
row
\end{minipage} & \begin{minipage}[t]{0.22\columnwidth}\raggedleft
5.0
\end{minipage} & \begin{minipage}[t]{0.35\columnwidth}\raggedright
Here's another one. Note the blank line between rows.
\end{minipage}
\\\noalign{\medskip}
\hline
\noalign{\medskip}
\caption{Here's the caption. In pandoc markdown it, too, may span multiple lines.}
\end{longtable}

\begin{fullwidth}
\lipsum[9]
\end{fullwidth}

\end{document}